https://www.visionlearning.com/en/library/Chemistry/1/Kinetic-Molecular-Theory/251

Modern KMT

A major use of modern KMT is as a framework for understanding gases and predicting their behavior. KMT links the microscopic behaviors of ideal gas molecules to the macroscopic properties of gases. In its current form, KMT makes five assumptions about ideal gas molecules:

        Gases consist of many molecules in constant, random, linear motion.
        The volume of all the molecules is negligible compared to the gas’s total volume.
        Intermolecular forces are negligible.
        The average kinetic energy of all molecules does not change, so long as the gas’s temperature is constant. In other words, collisions between molecules are perfectly elastic.
        The average kinetic energy of all molecules is proportional to the absolute temperature of the gas. This means that, at any temperature, gas molecules in equilibrium have the same average kinetic energy (but NOT the same velocity and mass).

With KMT’s assumptions, scientists are able to describe on a molecular level the behaviors of gases. These behaviors are common to all gases because of the relationships between gas pressure, volume, temperature, and amount, which are described and predicted by the gas laws (for more on the gas laws, please see our Properties of Gases module). But KMT and the gas laws are useful for understanding more than abstract ideas about chemistry. With KMT and the gas laws, we can better understand the behaviors of real gases, such as the air we use to inflate tires, as we’ll explore more below.
