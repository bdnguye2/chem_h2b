%%%%%%%%%%%%%%%%%%%%%%%%%%%%%%%%%%%%%%%%%%%%%%%%%%%%%%%%%%%%%%%%%%%%%%%%
% Preamble
%%%%%%%%%%%%%%%%%%%%%%%%%%%%%%%%%%%%%%%%%%%%%%%%%%%%%%%%%%%%%%%%%%%%%%%%
\documentclass[11pt]{article}
%
% Packages and other includes
% Pagination
\usepackage[letterpaper, margin=1in]{geometry}
\usepackage{emptypage}
%
% Fonts
\usepackage[T1]{fontenc} % best for Western European languages
\usepackage{lmodern} % Latin Modern instead of CM
\usepackage{textcomp} % required to get special symbols
%
% Math
\usepackage{amsmath, amssymb}
\usepackage{braket}
%
% Graphics, floats, tables
\usepackage{graphicx, color, float, array}
%
% Hyperlinks
\usepackage{hyperref}
%
%
% Definitions and settings
% Paragraph indent and spacing
\setlength{\parskip}{0.4\baselineskip}
\setlength{\parindent}{0in}
%
%
% Title, authors, date
\title{\textbf{Worksheet 2}}
\date{\vspace{-2em}January 11, 2022}
%
%
%%%%%%%%%%%%%%%%%%%%%%%%%%%%%%%%%%%%%%%%%%%%%%%%%%%%%%%%%%%%%%%%%%%%%%%%
% Main document
%%%%%%%%%%%%%%%%%%%%%%%%%%%%%%%%%%%%%%%%%%%%%%%%%%%%%%%%%%%%%%%%%%%%%%%%
%

\begin{document}

\maketitle

Weekly homework assignments are posted approximately one week prior to the
due date. Collaborations are encouraged and students must report all collaborators
in writing on each assignment. All external sources (websites, books) must be
properly cited. Additional problems are listed at the end of each assignment.
This week's assignment is due \textit{Tuesday, Jan 18th at 10:00am.}

\textbf{Ideal Gas Law}

1. (2 pts) What\ is the density (in g/L) of chloroform, CHCl$_3$, vapor at
$2.00\times 10^2\text{Torr}$ and $298\text{K}$. Report to 3 significant figures.

\vspace{1.5in}

2. (2 pts) A compound used in the manufacture of Saran is $24.7\%$ C, $2.10\%$ H, and
$73.2\%$ Cl by mass. The storage of $3.557\text{g}$ of the gaseous compound in
a $755$-mL vessel at $0^\circ\text{C}$ results in a pressure of $1.10\text{atm}$.
What is the molecular formula of the compound? Report to 3 significant figures.

% C2H2Cl2

\vspace{1.5in}

3. (2 pts) A vessel of volume $22.4\text{L}$ contains $2.00$ mol H$_2$(g) and $1.00$ mol
N$_2$(g) at $273.15\text{K}$. Calculate the partial pressures and the total pressure.
Report to 3 significant figures.

\vspace{1.7in}

4. (7 pts) A flask of volume 5.00L is evacuated and 43.78g of solid dinitrogen tetroxide,
N$_2$O$_4$, is introduced at $-196^\circ\text{C}$. The sample is then warmed to
$25^\circ\text{C}$, during which time the N$_2$O$_4$ vaporizes and some of it dissociates
to form brown NO$_2$ gas. The pressure slowly increases until it stabilizes at 2.96atm.
Report to 3 significant figures.

(a) Write a balanced equation for the reaction.

(b) If the gas in the flask at $25^\circ\text{C}$ were all N$_2$O$_4$, what would the pressure
be?

(c) If all the gas in the flask converted into NO$_2$, what would the pressure be?

(d) What are the mole fractions of N$_2$O$_4$ and NO$_2$ once the pressure stabilizes at
2.96atm?

\vspace{3in}

\textbf{First Law of Thermodynamics}

5. (4 pts) Calculate the work for each of the following processes beginning with a gas
sample in a piston assembly with $T=305\text{K}$, $P=1.79\text{atm}$, and
$V=52.9\text{L}$ by two different pathways. Report to 3 significant figures.

(a) Isothermal, reversible expansion to a final volume of $6.52\text{L}$.

(b) Irreversible expansion against a constant external pressure of $1.00\text{atm}$
to a final volume of $6.52\text{L}$.

\vfill
\textbf{Optional Additional Problems:} Ch. 10 - odd problems 95 - 117

\end{document}
