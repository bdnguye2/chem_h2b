%%%%%%%%%%%%%%%%%%%%%%%%%%%%%%%%%%%%%%%%%%%%%%%%%%%%%%%%%%%%%%%%%%%%%%%%
% Preamble
%%%%%%%%%%%%%%%%%%%%%%%%%%%%%%%%%%%%%%%%%%%%%%%%%%%%%%%%%%%%%%%%%%%%%%%%
\documentclass[11pt]{article}
%
% Packages and other includes
% Pagination
\usepackage[letterpaper, margin=1in]{geometry}
\usepackage{emptypage}
%
% Fonts
\usepackage[T1]{fontenc} % best for Western European languages
\usepackage{lmodern} % Latin Modern instead of CM
\usepackage{textcomp} % required to get special symbols
%
% Math
\usepackage{amsmath, amssymb}
\usepackage{braket}
%
% Graphics, floats, tables
\usepackage{graphicx, color, float, array}
%
% Hyperlinks
\usepackage{hyperref}
%
%
% Definitions and settings
% Paragraph indent and spacing
\setlength{\parskip}{0.4\baselineskip}
\setlength{\parindent}{0in}
%
%
% Title, authors, date
\title{\textbf{Midterm Problems}}
%
%
%%%%%%%%%%%%%%%%%%%%%%%%%%%%%%%%%%%%%%%%%%%%%%%%%%%%%%%%%%%%%%%%%%%%%%%%
% Main document
%%%%%%%%%%%%%%%%%%%%%%%%%%%%%%%%%%%%%%%%%%%%%%%%%%%%%%%%%%%%%%%%%%%%%%%%
%

\begin{document}

\maketitle

1. \textbf{Barometric Formula} The barometric formula is given by
\begin{equation*}
  P_h = P_0 e^{-\frac{Mgh}{RT}}
\end{equation*}
where $P_h$ is the pressure at height $h$, $P_0$ is the pressure at ground level,
$M$ is the molar mass of air (28.97 g/mol), $R$ is the gas constant, and $T$ is the
temperature. This formula has been used to approximate the elevation of mountains.
Report to 3 significant figures.

(a) A hiker brings a mercury barometer to measure the height of Mount Everest. At the
summit, the hiker reports the barometric pressure to be 253.0 Torr at $-9^\circ\text{C}$.
Use the barometric formula to approximate the height of Mount Everest.

% 8,503 meters calculated

(b) Mount Everest has an official height of 8,485 meters. Is the calculated height in
(a) overestimated or underestimated? Explain potential errors.

% Overestimated

(c) Given the barometric pressure in (a), compute the partial pressure of O$_2$(g) at the summit
(P$_{O2}$) assuming that the atmosphere is made of $21\%$ O$_2$. With the oxyhemoglobin dissociation
curve, estimate the percent hemoglobin saturated with O$_2$ assuming that the P$_{\text{O}2}$ in the blood
is equivalent to the P$_{\text{O}2}$ at the summit.

% https://breathe.ersjournals.com/content/11/3/194

\begin{center}
  \includegraphics[scale=0.23]{hbo2.png}
\end{center}

%
% https://penelope.uchicago.edu/Thayer/E/Journals/ISIS/12/3/Determinations_of_Heights_of_Mountains*.html
%
% barometric pressure - 242 mmHg or 0.318 atm; P_O2 = 0.0668 atm or 50.8 mmHg
%
%A hiker brings a mercury
%barometer to measure the height of Mount Everest. At the summit, the hiker reports the
%barometric pressure to be 253.0 Torr at $-9^\circ\text{C}$. Use the derived barometric
%formula to approximate the height of Mount Everest. Report to 4 significant figures.
%
% Deriving the Barometric Formula online
%
% https://www.youtube.com/watch?v=anAMD_KeB0s

2. \textbf{Isothermal Compression} Suppose 1.87 moles of Cl$_2$(g) at $35^\circ\text{C}$ are
compressed isothermally from a volume of 15.0L to 4.79L. Report to 3 significant figures.

(a) Sketch the process on the $PV$ diagram. Define all variables and show what corresponds
to the work ($w$) done on the gas

(b) Compute the work ($w$) and the heat ($q$) in kJ/mol.

(c) What is the final pressure of the gas?

3. \textbf{Decomposition of N$_2$O$_4$(g)} Supposed a sample of N$_2$O$_4$(g) has
a pressure of 6.6 kPa. After some time, a portion of it decomposes to form NO$_2$(g).
The total pressure of the mixture of gases is then 9.8 kPa. Assume the volume and the
temperature do not change. What percentage of N$_2$O$_4$(g) has decomposed? Report to
3 significant figures.

\pagebreak

\end{document}
