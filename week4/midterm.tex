%%%%%%%%%%%%%%%%%%%%%%%%%%%%%%%%%%%%%%%%%%%%%%%%%%%%%%%%%%%%%%%%%%%%%%%%
% Preamble
%%%%%%%%%%%%%%%%%%%%%%%%%%%%%%%%%%%%%%%%%%%%%%%%%%%%%%%%%%%%%%%%%%%%%%%%
\documentclass[11pt]{article}
%
% Packages and other includes
% Pagination
\usepackage[letterpaper, margin=1in]{geometry}
\usepackage{emptypage}
%
% Fonts
\usepackage[T1]{fontenc} % best for Western European languages
\usepackage{lmodern} % Latin Modern instead of CM
\usepackage{textcomp} % required to get special symbols
%
% Math
\usepackage{amsmath, amssymb}
\usepackage{braket}
%
% Graphics, floats, tables
\usepackage{graphicx, color, float, array}
%
% Hyperlinks
\usepackage{hyperref}
%
%
% Definitions and settings
% Paragraph indent and spacing
\setlength{\parskip}{0.4\baselineskip}
\setlength{\parindent}{0in}
%
%
% Title, authors, date
\title{\textbf{Midterm Problems}}
%
%
%%%%%%%%%%%%%%%%%%%%%%%%%%%%%%%%%%%%%%%%%%%%%%%%%%%%%%%%%%%%%%%%%%%%%%%%
% Main document
%%%%%%%%%%%%%%%%%%%%%%%%%%%%%%%%%%%%%%%%%%%%%%%%%%%%%%%%%%%%%%%%%%%%%%%%
%

\begin{document}

\maketitle

1. \textbf{Barometric Formula} The barometric formula is given by
\begin{equation*}
  P_h = P_0 e^{-\frac{Mgh}{RT}}
\end{equation*}
where $P_h$ is the pressure at height $h$, $P_0$ is the pressure at ground level,
$M$ is the molar mass of air (28.97 g/mol), $R$ is the gas constant, and $T$ is the
temperature. This formula has been used to approximate the elevations of mountains.

(a) A hiker brings a mercury barometer to measure the height of Mount Everest. At the
summit, the hiker reports the barometric pressure to be 253.0 Torr at $-9^\circ\text{C}$.
Use the derived barometric formula to approximate the height of Mount Everest.

(b) Mount Everest has an official height of 8,485 meters. Is the calculated height in
(a) overestimated or underestimated? Explain potential errors.

(c) Given the barometric pressure in (a), compute the partial pressure of O$_2$(g) assuming
that the atmosphere is made of $20\%$ O$_2$. Given the oxyhemoglobin dissociation curve,
estimate the percent hemoglobin saturated with O$_2$.

\begin{center}
  \includegraphics[scale=0.24]{hbo2.png}
\end{center}

%
% https://penelope.uchicago.edu/Thayer/E/Journals/ISIS/12/3/Determinations_of_Heights_of_Mountains*.html
%
% barometric pressure - 242 mmHg or 0.318 atm; P_O2 = 0.0668 atm or 50.8 mmHg
%
%A hiker brings a mercury
%barometer to measure the height of Mount Everest. At the summit, the hiker reports the
%barometric pressure to be 253.0 Torr at $-9^\circ\text{C}$. Use the derived barometric
%formula to approximate the height of Mount Everest. Report to 4 significant figures.
%
% Deriving the Barometric Formula online
%
% https://www.youtube.com/watch?v=anAMD_KeB0s

\pagebreak

\end{document}
